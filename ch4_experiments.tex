\chapter{Experiments}

In this chapter we present our experimental results and compare them to baseline solutions.
We perform the evaluation on multiple classification tasks. 

\section{Classification datasets}
    
    Our datasets were acquired by following instructions in the SentEval (\url{https://github.com/facebookresearch/SentEval}) repository.
    SentEval is a library for evaluating the quality of sentence embeddings \cite{conneau2017supervised}. 
    However it focuses on evaluating unsupervised embeddings.
    We have not used any code from this library, because we were not able to modify its API in a maintainable way to fit our needs. 
    In the end we just used it as a guideline for datasets acquisition.
    
    \subsection{Overview}
    We use a set of binary classification tasks that covers various types of sentence classification.
    Datasets corresponding to these tasks are:
    \begin{itemize}
        \item Movie review sentiment analysis MR.
        \item Product review dataset CR.
        \item Subjectivity/objectivity dataset SUBJ.
        \item Question-type dataset TREC.
        \item Opinion polarity dataset MPQA.
    \end{itemize}
    
\* % TODO remove first hline in code
\* %TODO add basic trec dataset with N/A

\begin{table}[h]
\begin{center}

\begin{tabular}{l|rrrr} 
\toprule
  &        CR &      MPQA &         MR &      SUBJ \\
\midrule
\#examples                             &   3775 &  10606 &   10662 &   10000 \\\hline 
\#unique words                         &   5674 &   6238 &   20325 &   22636 \\\hline 
\#words                                &  75932 &  32779 &  230162 &  246015 \\\hline 
\specialcell{avg sentence\\length}    &     20.11 &      3.09 &      21.59 &      24.6 \\\hline
\specialcell{max sentence\\length}    &    106 &     44 &      62 &     122 \\\hline 
\specialcell{median sentence\\length} &     18 &      2 &      21 &      23 \\\hline 
bias                                  &      0.64 &      0.69 &       0.50 &       0.5 \\
\bottomrule
\end{tabular}

\caption[Dataset characteristics]{Dataset characteristics}
\label{tab:datasets:stats}
\end{center}
\end{table}


    In table \ref{tab:datasets:stats} we present a few basic statistics for each dataset.
    Column \emph{bias} denotes percentage of the majority class in the dataset.
    It also presents a baseline accuracy which a naive algorithm could achieve by making constant predictions. 
    
    We discuss each of these datasets in more details and we provide some example sentences from these datasets.
    Note, that these examples  are displayed in very genuine. We see these examples as they really are and as they are presented to our algorithms.
    The actual preprocessing will be discussed later in section \ref{sec:preprocessing}.
    
    \subsubsection{Movie review, MR}
    
    With rise of internet forums and critique websites arose a problem of sentiment classification.
    
    This dataset contains movie reviews from site Rotten Tomatoes collected by Bo Pang and Lillian Lee \cite{pang2002thumbs}.
    Originally each comment is accompanied by categorical rating with values ``fresh'' (good movie) to ``rotten'' (bad movie). 
    These labels were transformed into this freely available dataset\footnote{\url{http://www.cs.cornell.edu/people/pabo/movie-review-data/}}.
    
    Example of positive review in this dataset is: \emph{the rock is destined to be the 21st century 's new conan and that he 's going to make a splash even greater than arnold schwarzenegger , jean-claud van damme or steven segal .}

    Example of an negative review: \emph{simplistic , silly and tedious .}
    
    \subsubsection{Product review, CR}
    
    This dataset was introduced by Hu and Liu \cite{hu2004mining} as a part of a customer review summarization task. \* % write more, chceck this citation
    They created a pipeline with a summarization pipeline that consists of: mining product features that have been commented on by customers; identifying opinion sentences in each review and deciding whether each opinion sentence is positive or negative;  summarizing  the  results.  
    
    We are only interested in the sentiment prediction part.
    
    Example of a positive review:
    \emph{im a more happier person after discovering the i/p button ! .}

    Example of a negative review:
    \emph{weaknesses are minor : the feel and layout of the remote control are only so-so ; . it does n 't show the complete file names of mp3s with really long names ; . you must cycle through every zoom setting ( 2x , 3x , 4x , 1/2x , etc . ) before getting back to normal size [ sorry if i 'm just ignorant of a way to get back to 1x quickly ] .}

    \subsubsection{Opinion polarity MPQA}

    Another sentiment dataset collected by Wiebe \cite{wiebe2005annotating}.
    Its focus is to capture not only the overall tone of the document but also strength of the tone.
    However we will use only the binary positive-negative classification.
    Wiebe designed a fine grained annotation scheme and employed it on sentence corpus of articles from the world press.
    
    Example of a positive sentence:
    \emph{are also being encouraged}
    
    Example of a negative sentence:
    \emph{complaining}  \* %TODO change to something better

    
    \subsubsection{Question-type, TREC}
    
    An important step in question answering and other dialog systems is to classify the  question to the anticipated type of the answer. 
    For example, the question of \emph{Who is a good boy?} should be classified into the type of animal (entity), because the question probably refers to some dog.  
    This information would narrow down the search space to identify the correct answer string \cite{huang2008question}. 
    
    This dataset contains questions and their labels in such classification.
    There are $6$ labels: 
    
    \begin{itemize}
        \item abbreviation (ABR): \emph{what is the full form of .com?}
        \item description (DESC): \emph{how did serfdom develop in and then leave russia?}
        \item entity (ENTY): \emph{what films featured the character popeye doyle?}
        \item human (HUM): \emph{what contemptible scoundrel stole the cork from my lunch?}
        \item location (LOC): \emph{what sprawling u.s. state boasts the most airports?}
        \item numeric (NUM): \emph{when was ozzy osbourne born?}
    \end{itemize}
    
    As can be seen, this dataset is not binary. 
    For our purposes, we create $6$ different binary datasets out of this one by employing \emph{one-vs-all} strategy (one-vs-the-rest) used while training prediction models.
    This strategy consists in fitting one classifier per class. 
    For each classifier, the class is fitted against all the other classes. 
    Hence new dataset TREC-ABR will contain the same samples, but labels will be binary, $1$ if the sentence was originally in class ABR and $0$ otherwise.
    
    \* % TODO add table for trec 


    \subsubsection{Subjectivity/objectivity, SUBJ}
    
    There is number of sub-tasks that can help in context of sentiment prediction.
    One such task is to decide whether the text was subjective or objective.
        
    Pang and Lee were able to mine the Web and create a large, automatically labeled sentence corpus. 
    To gather subjective sentences they collected movie review snippets from website  \url{www.rottentomatoes.com}.
    To obtain objective data, they took sentences from plot summaries available from the Internet Movie Database (\url{www.imdb.com}) \cite{pang2004sentimental} \footnote{We personally think, that this technique is rather questionable at best.}.
    
    Example of a objective review:
    \emph{the movie begins in the past where a young boy named sam attempts to save celebi from a hunter .}

    Example of a subjective review:
    \emph{smart and alert , thirteen conversations about one thing is a small gem .}
    

    \subsubsection{Preprocessing} \label{sec:preprocessing}
    
    Examples provided for each dataset could be slightly strange at the first glance.
    We deliberately did only minimal processing in terms of word filtering and normalization.
    We lower-case and tokenize the text. \* % TODO check the tokenization
    We do not filter any stop words (\texttt{the}) or non token characters (like question marks \texttt{?}). 
    We only filter words that are present in the corpus less then $10$ times.
    Not filtering words that are generally filtered showed to be an important decision that allowed us to get an insight from data.
    
    LSA is commonly accompanied by heavy filtering and it is known, that stop words removal helps the performance.
    Our rational behind minimal filtering is, that we would like the weighting scheme to understand, which words are important.
    Hence we hope, that if the stop word is really not important, its weight will be reduced.
    
    \subsection{Metrics}
    
    Because we deal with binary classification tasks, our metric of choice is an accuracy.
    
    Accuracy of our $p$ predictions $\hat{y}$ given the true labels $y$ can be computed as
    $$acc_y(\hat{y}) = \frac{1}{p}\sum_{i=1}^py_i ==\hat{y}_i$$
    
    Note that this metric can have problems with unbalanced datasets. 
    We just need to remember, that what is a good accuracy dpeends on the dataset.
    For example accuracy $0.69$ on MPQA dataset is really bad, because this accuracy can be achieved by constant classifier.
    In later experiments we will refer to the row with biases in table \ref{tab:datasets:stats}.
    
    We will compute accuracy on tree subsets of the dataset. 
    $\mathrm{acc}_{train}$ on the train set, $\mathrm{acc}_{valid}$ on the validation set and $\mathrm{acc}_{test}$ on the test set.
    Note that $\mathrm{acc}_{train}$ is not wery interesting on its own, but if we compare it with $\mathrm{acc}_{valid}$ and $\mathrm{acc}_{test}$ we can get an insight about whether our model is overfitting or underfitting.

\section{Results}
    In this
    
    \subsection{Baselines}
    
    
    
    \subsection{Our approach}
    

\section{Interpretability}
    \cite{ribeiro2016should} % explainability


