\chapter*{Introduction}
\addcontentsline{toc}{chapter}{Introduction}

The problem of document classification attracted a lot of attention recently. 
The most popular approaches are based on neural networks and neural word embeddings.
We revisit famous co-occurrence-based approach, the Latent Semantic Analysis.
This approach may perform poorly on document classification tasks, because it selects the most representative features and not the most discriminative ones. 
There was a number of attempts to add the supervised information into LSA and into its SVD part.
We build on the top of one of these approaches -- supervised weighting schemes.

\bigskip
Our goals are to propose a novel approach for learning task-specific word weights, explore performance of this approach and compare it with commonly used weighting schemes.
\bigskip

The main part of this thesis introduces the design of eLSA, approach that employs gradient descent technique to learn task specific supervised weights.
Further, we present thorough performance evaluation of eLSA on multiple datasets.
Finally, we use eLSA to gain a valuable insight about the classification tasks we want to solve and about commonly used weighting schemes.

\bigskip

In the first chapter, we introduce the problem of document classification and we describe some of the commonly used approaches to this problem.
In the second chapter, we present an overview of published literature about introducing supervision into the LSA.
In the third chapter, we introduce our novel approach eLSA, which builds on the top of supervised weights and LSA.
In the fourth chapter, we present results achieved by the eLSA and possible ways how to extract insight from its parameters. 



