\chapter*{Introduction}
\addcontentsline{toc}{chapter}{Introduction}

Recently the problem of document classification attracted a lot of attention. 
The most popular approach are based on neural networks and neural word embeddings.
We revisit a once very popular co-occurrence-based approach, the Latent Semantic Analysis.
This approach may perform poorly on document classification tasks, because it selects the most representative features and not the most discriminative ones. 
There was a number of attempts to add the supervised information into LSA and into its SVD part.
We build on top of one of these approaches, supervised weighting schemes.

\bigskip
Our goals are to propose a novel approach for leaning task-specific word weight, explore performance of this approach and compare it with commonly used weighting schemes.
\bigskip

The main part of this thesis deals with the design of eLSA, approach that employs gradient descent technique to learn task specific supervised weights.
Further we present performance evaluation of eLSA and we use it to gain valuable insight about the classification tasks we want to solve and about commonly used weighting schemes.

\bigskip

In the first chapter we introduce the problem of document classification and we describe some of the commonly used approaches for this problem.
In the second chapter we present an overviews of published literature that deals with introducing supervision to LSA.
In the third chapter we preset our novel approach eLSA, that builds on top of supervised weights and LSA.
In the fourth chapter we present results achieved by eLSA and possible ways how to extract insight from its parameters. 



