\documentclass[12pt, oneside]{book}
\usepackage[a4paper,top=2.5cm,bottom=2.5cm,left=3.5cm,right=2cm]{geometry}
\usepackage[utf8]{inputenc}
\usepackage[T1]{fontenc}
\usepackage{graphicx}
\usepackage{url}
\usepackage[english]{babel} % vypnite pre prace v anglictine
\usepackage{color}
\usepackage{stmaryrd}
\usepackage{amsmath} 
\usepackage{amsthm}
\usepackage{listings}
\usepackage{caption}
\usepackage{subcaption}
\usepackage{float}
\usepackage[final]{pdfpages} % kill final
\usepackage[none]{hyphenat}
\usepackage{algorithm2e}
\usepackage{hyperref}
\usepackage{booktabs}
\usepackage{indentfirst}
\usepackage{enumerate}

%\usepackage{epigraph}

\usepackage{nameref}


\definecolor{lightgray}{gray}{0.9}
\lstset{
    showstringspaces=false,
    basicstyle=\ttfamily,
    keywordstyle=\color{blue},
    commentstyle=\color[grey]{0.6},
    stringstyle=\color[RGB]{255,150,75}
}

\newcommand{\inlinecode}[2]{\colorbox{lightgray}{\lstinline[language=#1]$#2$}}

\DeclareCaptionType{equ}[][]

\lstset{
        mathescape=true,
        literate=
               {=}{$\leftarrow{}$}{1}
               {==}{$={}$}{1},
        morekeywords={if,then,else,return,for,def}
        }


\theoremstyle{definition}
\newtheorem{definition}{Definition}[section]


\linespread{1.25} % hodnota 1.25 by mala zodpovedat 1.5 riadkovaniu
\setcounter{secnumdepth}{3}

% -------------------
% --- Definicia zakladnych pojmov
% --- Vyplnte podla vasho zadania
% -------------------
\def\mfrok{2018}
\def\mfnazov{Improving LSA word weights for document classification}
\def\mftyp{master's thesis}
\def\mfautor{Bc. Vladimír Macko}
\def\mfskolitel{RNDr. Kristína Malinovská, PhD.}
\def\*{{\color{red} \bf FIXME: }}

%ak mate konzultanta, odkomentujte aj jeho meno na titulnom liste
\def\mfkonzultant{RNDr. Radim Řehůřek, PhD.}  

\def\mfmiesto{Bratislava, \mfrok}

%aj cislo odboru je povinne a je podla studijneho odboru autora prace
\def\mfodbor{ 2508 Informatics } 
\def\program{ Informatics }
\def\mfpracovisko{ Department of Computer Science }

\newcommand{\specialcell}[2][c]{%
  \begin{tabular}[#1]{@{}l@{}}#2\end{tabular}}

\begin{document}     

% -------------------
% --- Obalka ------
% -------------------
\thispagestyle{empty}

\begin{center}
\sc\large
COMENIUS UNIVERSITY, BRATISLAVA\\
FACULTY OF MATHEMATICS, PHYSICS AND INFORMATICS

\vfill

{\LARGE\mfnazov}\\
\mftyp
\end{center}

\vfill

{\sc\large 
\noindent \mfrok\\
\mfautor
}

\eject % EOP i
% --- koniec obalky ----

% -------------------
% --- Titulný list
% -------------------

\thispagestyle{empty}
\noindent

\begin{center}
\sc  
\large
COMENIUS UNIVERSITY, BRATISLAVA\\
FACULTY OF MATHEMATICS, PHYSICS AND INFORMATICS

\vfill

{\LARGE\mfnazov}\\
\mftyp
\end{center}

\vfill

\noindent
\begin{tabular}{ll}
Study programme: & \program \\
Study field: & \mfodbor \\
Department: & \mfpracovisko \\
Supervisor: & \mfskolitel \\
Consultant: & \mfkonzultant \\
\end{tabular}

\vfill


\noindent \mfmiesto\\
\mfautor

\eject % EOP i


% --- Koniec titulnej strany


% -------------------
% --- Zadanie z AIS
% -------------------
% v tlačenej verzii s podpismi zainteresovaných osôb.
% v elektronickej verzii sa zverejňuje zadanie bez podpisov

\newpage 
\thispagestyle{empty}
\includepdf[pages={1}]{misc/vmacko-zadanie-sk.PDF}

\newpage 
\thispagestyle{empty}
\includepdf[pages={1}]{misc/vmacko-zadanie-en.PDF}

% --- Koniec zadania

\frontmatter

% -------------------
%   Poďakovanie - nepovinné
% -------------------
\setcounter{page}{3}
\newpage 
~

\vfill
{\bf Acknowledgment:} \*
% I would like to thank Kika, 
% gf: 
% fellas from CEAI without skill acquired there i would not ba able to write this shit
% Radim rehured, who responded to email of a random guy that asked him for mentoring/advice
% Guido van Rossum-> Python tool that caused more joy and suffering at the same time
% A samozrejme, Basi, rodine a kamarátom za to, že to so mnou v poslednej dobe prežili a dokonca ma aj podporovali
% --- Koniec poďakovania




% -------------------
%   Abstrakt - Slovensky
% -------------------
\newpage 
\section*{Abstrakt}

Latentná sémantická analýza môže zlyhávať pri klasifikačných úlohách, lebo vyberá črty dokumentov, ktoré sú najreprezentatívnejšie ale nie najdiskriminatívnejšie. 
V práci predstavujeme novú metódu eLSA, ktorá prináša ďalšiu vrstvu váh $w'$, ktoré sú trénované pomocou metódy najväčšieho vzostupu.
Experimentálne sme ukázali, že proces učenia eLSA konverguje a že eLSA dosahuje väčšiu presnosť ako LSA.
Taktiež využívame eLSA na anylzovanie bežne používaných váhových schém a identifikujeme slová, ktoré tieto schémy podhodnocujú alebo nadhodnocujú.

\paragraph*{Kľúčové slová:} spracovanie prirodzeného jazyka, klasifikácia dokumentov, gradient descent, LSA
% --- Koniec Abstrakt - Slovensky


% -------------------
% --- Abstrakt - Anglicky 
% -------------------
\newpage 
\section*{Abstract}

Latent semantic analysis may perform poorly on document classification tasks, because it selects the most representative but not the most discriminative features.
We propose a new method, eLSA, that introduces another layer or weights $w'$, that are trained with gradient descent.
We experimentally show, that learning of eLSA converges, and that it achieves higher accuracy than LSA. 
We also use eLSA to analyze common weighting schemes and identify words, that these schemes underweight or overweight.

\paragraph*{Keywords:} natural language processing, document classification, gradient descent, LSA

% --- Koniec Abstrakt - Anglicky



% -------------------
% --- Predhovor - v informatike sa zvacsa nepouziva
% -------------------
%\newpage 
%\thispagestyle{empty}
%
%\huge{Predhovor}
%\normalsize
%\newline
%Predhovor je všeobecná informácia o práci, obsahuje hlavnú charakteristiku práce 
%a okolnosti jej vzniku. Autor zdôvodní výber témy, stručne informuje o cieľoch 
%a význame práce, spomenie domáci a zahraničný kontext, komu je práca určená, 
%použité metódy, stav poznania; autor stručne charakterizuje svoj prístup a svoje 
%hľadisko. 
%
% --- Koniec Predhovor


% -------------------
% --- Obsah
% -------------------

\newpage 

\tableofcontents

% ---  Koniec Obsahu

% -------------------
% --- Zoznamy tabuliek, obrázkov - nepovinne
% -------------------

%\newpage 

\listoffigures


\listoftables

% ---  Koniec Zoznamov

\mainmatter

\input text.tex 


\newpage	

{
    \backmatter
    
    \thispagestyle{empty}
    \nocite{*}
    \clearpage
    
    \bibliographystyle{plain}
    \bibliography{literatura} 
}

\appendix
\chapter{Source code}
\label{appendix:code}

Source code to this diploma thesis with installation guide is available on GitHub. 
\*% add link

Documentation also contains guide for extending the evaluation for further datasets, test different weighting schemes or different classifiers.

\chapter{Detailed results}
\label{appendix:detailed}

\section{BOW baselines}
\begin{table}[h]
\begin{center}

\begin{tabular}{llrrrr}
\toprule
{} &      &  CR &  MPQA &  MR &  SUBJ \\
scheme &  &            &              &            &              \\
\midrule
None & test &      0.787 &        0.838 &      0.761 &        0.907 \\
{} & train &      0.971 &        0.912 &      0.977 &        0.995 \\
tfchi2 & test &      0.751 &        0.806 &      0.684 &        0.836 \\
{} & train &      0.789 &        0.854 &      0.706 &        0.845 \\
tfgr & test &      0.747 &        0.813 &      0.668 &        0.840 \\
{} & train &      0.789 &        0.856 &      0.709 &        0.845 \\
tfidf & test &      0.771 &        0.838 &      0.747 &        0.904 \\
{} & train &      0.999 &        0.979 &      1.000 &        1.000 \\
tfig & test &      0.769 &        0.815 &      0.675 &        0.839 \\
{} & train &      0.790 &        0.854 &      0.712 &        0.847 \\
tfor & test &      0.792 &        0.838 &      0.769 &        0.908 \\
{} & train &      0.893 &        0.886 &      0.903 &        0.936 \\
tfrf & test &      0.762 &        0.827 &      0.728 &        0.884 \\
{} & train &      0.837 &        0.869 &      0.823 &        0.919 \\
\bottomrule
\end{tabular}

\caption[Accuracy for BOW baseline]{Accuracy for BOW baseline}
\label{}
\end{center}
\end{table}


\begin{table}[h]
\begin{center}

\begin{tabular}{llrrrrrr}
\toprule
{} &&  ABBR &  DESC &  ENTY &  HUM &  LOC &  NUM \\
scheme &  & & & &&&\\
\midrule
None & test & 0.995 & 0.925 & 0.874 &0.920 &0.960 &0.959 \\
{} & train & 0.995 & 0.981 & 0.983 &0.986 &0.987 &0.990 \\
tfchi2 & test & 0.993 & 0.878 & 0.789 &0.899 &0.945 &0.923 \\
{} & train & 0.993 & 0.872 & 0.787 &0.895 &0.949 &0.919 \\
tfgr & test & 0.991 & 0.866 & 0.784 &0.889 &0.942 &0.924 \\
{} & train & 0.991 & 0.878 & 0.788 &0.896 &0.946 &0.926 \\
tfidf & test & 0.991 & 0.925 & 0.885 &0.930 &0.966 &0.968 \\
{} & train & 1.000 & 1.000 & 1.000 &1.000 &1.000 &1.000 \\
tfig & test & 0.991 & 0.872 & 0.784 &0.887 &0.949 &0.922 \\
{} & train & 0.991 & 0.878 & 0.788 &0.896 &0.947 &0.924 \\
tfor & test & 0.990 & 0.864 & 0.852 &0.930 &0.952 &0.948 \\
{} & train & 0.991 & 0.954 & 0.940 &0.969 &0.969 &0.971 \\
tfrf & test & 0.989 & 0.855 & 0.837 &0.908 &0.930 &0.943 \\
{} & train & 0.991 & 0.926 & 0.906 &0.942 &0.951 &0.957 \\
\bottomrule
\end{tabular}

\caption[Accuracy for BOW baseline on TREC datasets]{Accuracy for BOW baseline on TREC datasets}
\label{}
\end{center}
\end{table}

\newpage
\section{LSA baselines}



\begin{table}[h]
\begin{center}

\begin{tabular}{llrrrr}
\toprule
{} &      &  CR &  MPQA &  MR &  SUBJ \\
scheme &  &            &              &            &              \\
\midrule
None & test &      0.753 &        0.744 &      0.662 &        0.871 \\
{} & train &      0.798 &        0.757 &      0.696 &        0.887 \\
tfchi2 & test &      0.749 &        0.770 &      0.666 &        0.833 \\
{} & train &      0.784 &        0.787 &      0.687 &        0.843 \\
tfgr & test &      0.754 &        0.772 &      0.671 &        0.831 \\
{} & train &      0.787 &        0.785 &      0.690 &        0.845 \\
tfidf & test &      0.753 &        0.750 &      0.683 &        0.890 \\
{} & train &      0.801 &        0.761 &      0.710 &        0.900 \\
tfig & test &      0.758 &        0.772 &      0.658 &        0.837 \\
{} & train &      0.785 &        0.785 &      0.692 &        0.845 \\
tfor & test &      0.780 &        0.776 &      0.729 &        0.878 \\
{} & train &      0.844 &        0.787 &      0.773 &        0.899 \\
tfrf & test &      0.753 &        0.782 &      0.676 &        0.874 \\
{} & train &      0.802 &        0.787 &      0.716 &        0.889 \\
\bottomrule
\end{tabular}

\caption[Accuracy for LSA baseline with 200 dimensions]{Accuracy for LSA baseline with 200 dimensions}
\label{tab:lsa:resuts:abs:200}
\end{center}
\end{table}





\begin{table}[h]
\begin{center}

\begin{tabular}{llrrrr}
\toprule
{} &      &  CR &  MPQA &  MR &  SUBJ \\
scheme &  &            &              &            &              \\
\midrule
None & test &      0.760 &        0.763 &      0.692 &        0.893 \\
{} & train &      0.822 &        0.783 &      0.721 &        0.900 \\
tfchi2 & test &      0.763 &        0.785 &      0.679 &        0.829 \\
{} & train &      0.786 &        0.803 &      0.696 &        0.843 \\
tfgr & test &      0.765 &        0.788 &      0.666 &        0.842 \\
{} & train &      0.787 &        0.802 &      0.698 &        0.845 \\
tfidf & test &      0.778 &        0.763 &      0.703 &        0.890 \\
{} & train &      0.831 &        0.788 &      0.739 &        0.912 \\
tfig & test &      0.751 &        0.780 &      0.668 &        0.839 \\
{} & train &      0.789 &        0.804 &      0.698 &        0.847 \\
tfor & test &      0.798 &        0.781 &      0.744 &        0.892 \\
{} & train &      0.854 &        0.805 &      0.792 &        0.906 \\
tfrf & test &      0.776 &        0.783 &      0.691 &        0.867 \\
{} & train &      0.809 &        0.802 &      0.733 &        0.895 \\
\bottomrule
\end{tabular}

\caption[Accuracy for LSA baseline with 300 dimensions]{Accuracy for LSA baseline with 300 dimensions}
\label{tab:lsa:resuts:abs:300}
\end{center}
\end{table}





\begin{table}[h]
\begin{center}

\begin{tabular}{llrrrr}
\toprule
{} &      &  CR &  MPQA &  MR &  SUBJ \\
scheme &  &            &              &            &              \\
\midrule
None & test &      0.752 &        0.775 &      0.712 &        0.887 \\
{} & train &      0.845 &        0.799 &      0.747 &        0.911 \\
tfchi2 & test &      0.760 &        0.791 &      0.680 &        0.834 \\
{} & train &      0.788 &        0.813 &      0.699 &        0.846 \\
tfgr & test &      0.754 &        0.790 &      0.680 &        0.827 \\
{} & train &      0.790 &        0.815 &      0.702 &        0.848 \\
tfidf & test &      0.767 &        0.784 &      0.731 &        0.892 \\
{} & train &      0.851 &        0.802 &      0.758 &        0.918 \\
tfig & test &      0.742 &        0.789 &      0.676 &        0.838 \\
{} & train &      0.790 &        0.813 &      0.702 &        0.846 \\
tfor & test &      0.793 &        0.792 &      0.756 &        0.879 \\
{} & train &      0.863 &        0.820 &      0.803 &        0.912 \\
tfrf & test &      0.765 &        0.795 &      0.706 &        0.871 \\
{} & train &      0.815 &        0.814 &      0.747 &        0.901 \\
\bottomrule
\end{tabular}

\caption[Accuracy for LSA baseline with 400 dimensions]{Accuracy for LSA baseline with 400 dimensions}
\label{tab:lsa:resuts:abs:400}
\end{center}
\end{table}





\begin{table}[h]
\begin{center}

\begin{tabular}{llrrrr}
\toprule
{} &      &  CR &  MPQA &  MR &  SUBJ \\
scheme &  &            &              &            &              \\
\midrule
None & test &      0.116 &        0.056 &      0.162 &        0.371 \\
{} & train &      0.160 &        0.069 &      0.196 &        0.387 \\
tfchi2 & test &      0.111 &        0.082 &      0.166 &        0.333 \\
{} & train &      0.147 &        0.099 &      0.187 &        0.343 \\
tfgr & test &      0.117 &        0.084 &      0.171 &        0.331 \\
{} & train &      0.149 &        0.098 &      0.190 &        0.345 \\
tfidf & test &      0.115 &        0.063 &      0.183 &        0.390 \\
{} & train &      0.164 &        0.073 &      0.210 &        0.400 \\
tfig & test &      0.120 &        0.085 &      0.158 &        0.337 \\
{} & train &      0.148 &        0.098 &      0.192 &        0.345 \\
tfor & test &      0.142 &        0.088 &      0.229 &        0.378 \\
{} & train &      0.207 &        0.099 &      0.273 &        0.399 \\
tfrf & test &      0.115 &        0.094 &      0.176 &        0.374 \\
{} & train &      0.165 &        0.099 &      0.216 &        0.389 \\
\bottomrule
\end{tabular}

\caption[Accuracy improvements for LSA baseline with 200 dimensions]{Accuracy improvements for LSA baseline with 200 dimensions}
\label{tab:lsa:resuts:200}
\end{center}
\end{table}





\begin{table}[h]
\begin{center}

\begin{tabular}{llrrrr}
\toprule
{} &      &  CR &  MPQA &  MR &  SUBJ \\
scheme &  &            &              &            &              \\
\midrule
None & test &      0.122 &        0.075 &      0.192 &        0.393 \\
{} & train &      0.185 &        0.095 &      0.221 &        0.400 \\
tfchi2 & test &      0.125 &        0.097 &      0.179 &        0.329 \\
{} & train &      0.149 &        0.115 &      0.196 &        0.343 \\
tfgr & test &      0.127 &        0.100 &      0.166 &        0.342 \\
{} & train &      0.149 &        0.114 &      0.198 &        0.345 \\
tfidf & test &      0.141 &        0.075 &      0.203 &        0.390 \\
{} & train &      0.193 &        0.100 &      0.239 &        0.412 \\
tfig & test &      0.113 &        0.092 &      0.168 &        0.339 \\
{} & train &      0.151 &        0.116 &      0.198 &        0.347 \\
tfor & test &      0.160 &        0.094 &      0.244 &        0.392 \\
{} & train &      0.216 &        0.117 &      0.292 &        0.406 \\
tfrf & test &      0.138 &        0.095 &      0.191 &        0.367 \\
{} & train &      0.172 &        0.114 &      0.233 &        0.395 \\
\bottomrule
\end{tabular}

\caption[Accuracy improvements for LSA baseline with 300 dimensions]{Accuracy improvements for LSA baseline with 300 dimensions}
\label{tab:lsa:resuts:300}
\end{center}
\end{table}





\begin{table}[h]
\begin{center}

\begin{tabular}{llrrrr}
\toprule
{} &      &  CR &  MPQA &  MR &  SUBJ \\
scheme &  &            &              &            &              \\
\midrule
None & test &      0.114 &        0.088 &      0.212 &        0.387 \\
{} & train &      0.207 &        0.111 &      0.247 &        0.411 \\
tfchi2 & test &      0.123 &        0.103 &      0.180 &        0.334 \\
{} & train &      0.150 &        0.125 &      0.199 &        0.346 \\
tfgr & test &      0.116 &        0.102 &      0.180 &        0.327 \\
{} & train &      0.152 &        0.127 &      0.202 &        0.348 \\
tfidf & test &      0.129 &        0.096 &      0.231 &        0.392 \\
{} & train &      0.213 &        0.114 &      0.258 &        0.418 \\
tfig & test &      0.105 &        0.102 &      0.176 &        0.338 \\
{} & train &      0.152 &        0.125 &      0.202 &        0.346 \\
tfor & test &      0.156 &        0.104 &      0.256 &        0.379 \\
{} & train &      0.226 &        0.132 &      0.303 &        0.412 \\
tfrf & test &      0.127 &        0.107 &      0.206 &        0.371 \\
{} & train &      0.178 &        0.127 &      0.247 &        0.401 \\
\bottomrule
\end{tabular}

\caption[Accuracy improvements for LSA baseline with 400 dimensions]{Accuracy improvements for LSA baseline with 400 dimensions}
\label{tab:lsa:resuts:400}
\end{center}
\end{table}






\begin{table}[h]
\begin{center}

\begin{tabular}{llrrrrrr}
\toprule
{} &  &  ABBR &  DESC &  ENTY &  HUM &  LOC &  NUM \\
scheme &  &       &       &       &      &      &      \\
\midrule
None & test &     0.992 &     0.900 &     0.841 &    0.912 &    0.946 &    0.949 \\
{} & train &     0.992 &     0.909 &     0.872 &    0.930 &    0.956 &    0.954 \\
tfchi2 & test &     0.992 &     0.866 &     0.785 &    0.889 &    0.948 &    0.915 \\
{} & train &     0.994 &     0.875 &     0.787 &    0.897 &    0.949 &    0.919 \\
tfgr & test &     0.991 &     0.877 &     0.787 &    0.895 &    0.942 &    0.933 \\
{} & train &     0.991 &     0.879 &     0.787 &    0.895 &    0.947 &    0.926 \\
tfidf & test &     0.992 &     0.890 &     0.853 &    0.914 &    0.949 &    0.948 \\
{} & train &     0.995 &     0.906 &     0.877 &    0.936 &    0.961 &    0.966 \\
tfig & test &     0.989 &     0.872 &     0.787 &    0.891 &    0.939 &    0.923 \\
{} & train &     0.991 &     0.881 &     0.788 &    0.893 &    0.947 &    0.921 \\
tfor & test &     0.992 &     0.864 &     0.850 &    0.916 &    0.948 &    0.956 \\
{} & train &     0.990 &     0.941 &     0.926 &    0.963 &    0.963 &    0.968 \\
tfrf & test &     0.990 &     0.856 &     0.831 &    0.903 &    0.941 &    0.943 \\
{} & train &     0.991 &     0.899 &     0.880 &    0.932 &    0.948 &    0.952 \\
\bottomrule
\end{tabular}

\caption[Accuracy for LSA baseline with 200 dimensions on TREC datasets]{Accuracy for LSA baseline with 200 dimensions on TREC datasets}
\label{tab:lsa:resuts:abs:200:TREC}
\end{center}
\end{table}





\begin{table}[h]
\begin{center}

\begin{tabular}{llrrrrrr}
\toprule
{} &  &  ABBR &  DESC &  ENTY &  HUM &  LOC &  NUM \\
scheme &  &       &       &       &      &      &      \\
\midrule
None & test &     0.992 &     0.906 &     0.847 &    0.917 &    0.953 &    0.955 \\
{} & train &     0.993 &     0.922 &     0.885 &    0.944 &    0.959 &    0.963 \\
tfchi2 & test &     0.994 &     0.884 &     0.787 &    0.895 &    0.944 &    0.921 \\
{} & train &     0.992 &     0.879 &     0.789 &    0.896 &    0.949 &    0.918 \\
tfgr & test &     0.992 &     0.869 &     0.786 &    0.886 &    0.944 &    0.926 \\
{} & train &     0.991 &     0.880 &     0.787 &    0.896 &    0.949 &    0.923 \\
tfidf & test &     0.993 &     0.903 &     0.859 &    0.925 &    0.951 &    0.954 \\
{} & train &     0.995 &     0.923 &     0.896 &    0.954 &    0.970 &    0.975 \\
tfig & test &     0.990 &     0.876 &     0.787 &    0.886 &    0.941 &    0.925 \\
{} & train &     0.991 &     0.879 &     0.787 &    0.895 &    0.948 &    0.930 \\
tfor & test &     0.990 &     0.865 &     0.857 &    0.924 &    0.955 &    0.957 \\
{} & train &     0.991 &     0.943 &     0.931 &    0.966 &    0.965 &    0.971 \\
tfrf & test &     0.990 &     0.865 &     0.832 &    0.910 &    0.942 &    0.949 \\
{} & train &     0.991 &     0.906 &     0.887 &    0.935 &    0.949 &    0.953 \\
\bottomrule
\end{tabular}

\caption[Accuracy for LSA baseline with 300 dimensions on TREC datasets]{Accuracy for LSA baseline with 300 dimensions on TREC datasets}
\label{tab:lsa:resuts:abs:300:TREC}
\end{center}
\end{table}





\begin{table}[h]
\begin{center}

\begin{tabular}{llrrrrrr}
\toprule
{} &  &  ABBR &  DESC &  ENTY &  HUM &  LOC &  NUM \\
scheme &  &       &       &       &      &      &      \\
\midrule
None & test &     0.992 &     0.913 &     0.865 &    0.920 &    0.943 &    0.956 \\
{} & train &     0.994 &     0.932 &     0.900 &    0.948 &    0.965 &    0.967 \\
tfchi2 & test &     0.993 &     0.879 &     0.785 &    0.895 &    0.945 &    0.915 \\
{} & train &     0.993 &     0.879 &     0.787 &    0.898 &    0.948 &    0.917 \\
tfgr & test &     0.987 &     0.875 &     0.786 &    0.891 &    0.945 &    0.925 \\
{} & train &     0.992 &     0.881 &     0.787 &    0.896 &    0.948 &    0.928 \\
tfidf & test &     0.994 &     0.905 &     0.859 &    0.923 &    0.953 &    0.957 \\
{} & train &     0.995 &     0.933 &     0.909 &    0.965 &    0.980 &    0.985 \\
tfig & test &     0.990 &     0.878 &     0.785 &    0.895 &    0.944 &    0.922 \\
{} & train &     0.991 &     0.880 &     0.787 &    0.896 &    0.947 &    0.923 \\
tfor & test &     0.993 &     0.867 &     0.847 &    0.928 &    0.949 &    0.959 \\
{} & train &     0.991 &     0.946 &     0.934 &    0.965 &    0.965 &    0.970 \\
tfrf & test &     0.991 &     0.863 &     0.832 &    0.917 &    0.944 &    0.946 \\
{} & train &     0.991 &     0.906 &     0.887 &    0.936 &    0.950 &    0.954 \\
\bottomrule
\end{tabular}

\caption[Accuracy for LSA baseline with 400 dimensions on TREC datasets]{Accuracy for LSA baseline with 400 dimensions on TREC datasets}
\label{tab:lsa:resuts:abs:400:TREC}
\end{center}
\end{table}





\begin{table}[h]
\begin{center}

\begin{tabular}{llrrrrrr}
\toprule
{} &  &  ABBR &  DESC &  ENTY &  HUM &  LOC &  NUM \\
scheme &  &       &       &       &      &      &      \\
\midrule
None & test &     0.008 &     0.119 &     0.067 &    0.129 &    0.100 &    0.119 \\
{} & train &     0.008 &     0.127 &     0.098 &    0.147 &    0.110 &    0.124 \\
tfchi2 & test &     0.008 &     0.085 &     0.010 &    0.105 &    0.102 &    0.085 \\
{} & train &     0.010 &     0.094 &     0.013 &    0.113 &    0.103 &    0.088 \\
tfgr & test &     0.007 &     0.096 &     0.013 &    0.111 &    0.096 &    0.102 \\
{} & train &     0.007 &     0.098 &     0.013 &    0.112 &    0.101 &    0.095 \\
tfidf & test &     0.008 &     0.108 &     0.078 &    0.131 &    0.103 &    0.117 \\
{} & train &     0.011 &     0.125 &     0.103 &    0.153 &    0.115 &    0.135 \\
tfig & test &     0.005 &     0.091 &     0.012 &    0.108 &    0.093 &    0.093 \\
{} & train &     0.007 &     0.100 &     0.014 &    0.109 &    0.101 &    0.091 \\
tfor & test &     0.008 &     0.083 &     0.076 &    0.133 &    0.102 &    0.126 \\
{} & train &     0.006 &     0.159 &     0.152 &    0.179 &    0.117 &    0.138 \\
tfrf & test &     0.006 &     0.074 &     0.057 &    0.120 &    0.095 &    0.113 \\
{} & train &     0.007 &     0.117 &     0.105 &    0.148 &    0.102 &    0.122 \\
\bottomrule
\end{tabular}

\caption[Accuracy improvements for LSA baseline with 200 dimensions on TREC datasets]{Accuracy improvements for LSA baseline with 200 dimensions on TREC datasets}
\label{tab:lsa:resuts:200:TREC}
\end{center}
\end{table}





\begin{table}[h]
\begin{center}

\begin{tabular}{llrrrrrr}
\toprule
{} &  &  ABBR &  DESC &  ENTY &  HUM &  LOC &  NUM \\
scheme &  &       &       &       &      &      &      \\
\midrule
None & test &     0.008 &     0.124 &     0.073 &    0.134 &    0.107 &    0.124 \\
{} & train &     0.009 &     0.140 &     0.110 &    0.160 &    0.113 &    0.132 \\
tfchi2 & test &     0.010 &     0.103 &     0.012 &    0.111 &    0.098 &    0.091 \\
{} & train &     0.008 &     0.098 &     0.015 &    0.112 &    0.103 &    0.087 \\
tfgr & test &     0.008 &     0.088 &     0.012 &    0.102 &    0.098 &    0.096 \\
{} & train &     0.007 &     0.098 &     0.012 &    0.113 &    0.103 &    0.093 \\
tfidf & test &     0.009 &     0.121 &     0.085 &    0.142 &    0.105 &    0.123 \\
{} & train &     0.011 &     0.142 &     0.121 &    0.170 &    0.124 &    0.144 \\
tfig & test &     0.006 &     0.095 &     0.013 &    0.102 &    0.095 &    0.094 \\
{} & train &     0.007 &     0.097 &     0.012 &    0.112 &    0.102 &    0.099 \\
tfor & test &     0.006 &     0.084 &     0.083 &    0.140 &    0.109 &    0.127 \\
{} & train &     0.007 &     0.162 &     0.156 &    0.182 &    0.119 &    0.140 \\
tfrf & test &     0.006 &     0.084 &     0.058 &    0.127 &    0.096 &    0.119 \\
{} & train &     0.007 &     0.124 &     0.113 &    0.151 &    0.103 &    0.122 \\
\bottomrule
\end{tabular}

\caption[Accuracy improvements for LSA baseline with 300 dimensions on TREC datasets]{Accuracy improvements for LSA baseline with 300 dimensions on TREC datasets}
\label{tab:lsa:resuts:300:TREC}
\end{center}
\end{table}





\begin{table}[h]
\begin{center}

\begin{tabular}{llrrrrrr}
\toprule
{} &  &  ABBR &  DESC &  ENTY &  HUM &  LOC &  NUM \\
scheme &  &       &       &       &      &      &      \\
\midrule
None & test &     0.008 &     0.132 &     0.091 &    0.136 &    0.097 &    0.125 \\
{} & train &     0.010 &     0.150 &     0.126 &    0.165 &    0.118 &    0.137 \\
tfchi2 & test &     0.009 &     0.097 &     0.010 &    0.111 &    0.099 &    0.084 \\
{} & train &     0.009 &     0.097 &     0.013 &    0.114 &    0.102 &    0.086 \\
tfgr & test &     0.003 &     0.094 &     0.012 &    0.107 &    0.099 &    0.094 \\
{} & train &     0.007 &     0.099 &     0.013 &    0.113 &    0.102 &    0.098 \\
tfidf & test &     0.010 &     0.124 &     0.085 &    0.140 &    0.107 &    0.127 \\
{} & train &     0.011 &     0.151 &     0.135 &    0.181 &    0.134 &    0.155 \\
tfig & test &     0.006 &     0.096 &     0.010 &    0.111 &    0.098 &    0.091 \\
{} & train &     0.007 &     0.098 &     0.013 &    0.112 &    0.101 &    0.093 \\
tfor & test &     0.009 &     0.086 &     0.073 &    0.144 &    0.103 &    0.129 \\
{} & train &     0.007 &     0.165 &     0.159 &    0.182 &    0.119 &    0.139 \\
tfrf & test &     0.007 &     0.082 &     0.058 &    0.134 &    0.098 &    0.115 \\
{} & train &     0.007 &     0.124 &     0.113 &    0.152 &    0.104 &    0.124 \\
\bottomrule
\end{tabular}

\caption[Accuracy improvements for LSA baseline with 400 dimensions on TREC datasets]{Accuracy improvements for LSA baseline with 400 dimensions on TREC datasets}
\label{tab:lsa:resuts:400:TREC}
\end{center}
\end{table}





\section{eLSA}

\begin{table}[h]
\begin{center}

\begin{tabular}{ll|rrrr}
\toprule
   &   &   CR &  MPQA &   MR &  SUBJ \\
scheme & lsa &        &        &        &        \\
\midrule
None & 200 &     -0.01 & \textbf{0.01} & \textbf{0.03} & \textbf{0.01} \\
   & 300 & \textbf{0.02} &     -0.0 & \textbf{0.03} &     -0.01 \\
   & 400 & \textbf{0.01} & \textbf{0.01} & \textbf{0.02} &     -0.0 \\
tfchi2 & 200 &     -0.01 &      0.0 &      0.0 & \textbf{0.01} \\
   & 300 &     -0.02 &      0.0 &     -0.0 & \textbf{0.01} \\
   & 400 &     -0.01 & \textbf{0.01} &     -0.0 &      0.0 \\
tfgr & 200 & \textbf{0.02} &     -0.0 & \textbf{0.02} & \textbf{0.01} \\
   & 300 &     -0.02 &     -0.0 & \textbf{0.03} & \textbf{0.01} \\
   & 400 &      0.0 & \textbf{0.01} &     -0.01 & \textbf{0.01} \\
tfidf & 200 &      0.0 & \textbf{0.01} & \textbf{0.05} &      0.0 \\
   & 300 &     -0.01 & \textbf{0.02} & \textbf{0.03} & \textbf{0.01} \\
   & 400 &      0.0 & \textbf{0.01} & \textbf{0.02} & \textbf{0.01} \\
tfig & 200 & \textbf{0.02} &     -0.01 & \textbf{0.02} & \textbf{0.01} \\
   & 300 &     -0.0 &      0.0 & \textbf{0.01} &     -0.0 \\
   & 400 & \textbf{0.02} & \textbf{0.01} & \textbf{0.01} &      0.0 \\
tfor & 200 &      0.0 &     -0.0 & \textbf{0.01} & \textbf{0.01} \\
   & 300 &     -0.02 & \textbf{0.02} &     -0.02 &      0.0 \\
   & 400 &     -0.01 &      0.0 &     -0.01 & \textbf{0.02} \\
tfrf & 200 &     -0.01 &     -0.01 &      0.0 &     -0.0 \\
   & 300 &     -0.0 &      0.0 &     -0.0 &      0.0 \\
   & 400 &     -0.0 & \textbf{0.01} &      0.0 &      0.0 \\
\bottomrule
\end{tabular}

\caption[Accuracy increase over LSA for $\alpha=0.01$]{Accuracy increase over LSA for $\alpha=0.01$}
\label{tab:batch:results0.01}
\end{center}
\end{table}






\begin{table}[h]
\begin{center}

\begin{tabular}{ll|rrrrrr}
\toprule
   &   & ABBR & DESC & ENTY & HUM & LOC & NUM \\
scheme & lsa &         &         &         &         &         &         \\
\midrule
None & 200 &       -0.0 &  \textbf{0.02} &  \textbf{0.01} &       0.0 &  \textbf{0.01} &      -0.0 \\
   & 300 &       0.0 &       -0.0 &  \textbf{0.01} &      -0.01 &       0.0 &      -0.0 \\
   & 400 &       -0.0 &       0.0 &       -0.0 &      -0.01 &  \textbf{0.02} &      -0.01 \\
tfchi2 & 200 &       0.0 &  \textbf{0.01} &       0.0 &  \textbf{0.01} &  \textbf{0.01} &  \textbf{0.01} \\
   & 300 &       -0.0 &       -0.0 &  \textbf{0.01} &       0.0 &       0.0 &  \textbf{0.01} \\
   & 400 &       0.0 &      -0.01 &  \textbf{0.01} &  \textbf{0.01} &       0.0 &  \textbf{0.01} \\
tfgr & 200 &       -0.0 &       0.0 &       0.0 &      -0.0 &  \textbf{0.01} &  \textbf{0.01} \\
   & 300 &       -0.0 &  \textbf{0.03} &       -0.0 &  \textbf{0.01} &      -0.0 &       0.0 \\
   & 400 &  \textbf{0.01} &  \textbf{0.01} &       -0.0 &  \textbf{0.01} &  \textbf{0.01} &       0.0 \\
tfidf & 200 &       0.0 &  \textbf{0.01} &       -0.0 &  \textbf{0.02} &       0.0 &  \textbf{0.01} \\
   & 300 &       0.0 &      -0.01 &  \textbf{0.01} &      -0.01 &       0.0 &  \textbf{0.01} \\
   & 400 &       -0.0 &       -0.0 &      -0.01 &       0.0 &  \textbf{0.01} &      -0.0 \\
tfig & 200 &       0.0 &  \textbf{0.02} &       0.0 &  \textbf{0.01} &       0.0 &       0.0 \\
   & 300 &       0.0 &       0.0 &       0.0 &  \textbf{0.01} &      -0.0 &      -0.0 \\
   & 400 &       0.0 &      -0.01 &       -0.0 &      -0.0 &      -0.0 &  \textbf{0.01} \\
tfor & 200 &       0.0 &  \textbf{0.01} &       -0.0 &  \textbf{0.01} &       0.0 &      -0.0 \\
   & 300 &       0.0 &  \textbf{0.02} &       -0.0 &      -0.0 &       0.0 &      -0.01 \\
   & 400 &       0.0 &  \textbf{0.05} &  \textbf{0.01} &      -0.0 &  \textbf{0.01} &      -0.01 \\
tfrf & 200 &       0.0 &  \textbf{0.04} &  \textbf{0.01} &  \textbf{0.01} &       0.0 &  \textbf{0.01} \\
   & 300 &       0.0 &  \textbf{0.05} &  \textbf{0.01} &       0.0 &      -0.0 &      -0.0 \\
   & 400 &       0.0 &  \textbf{0.04} &  \textbf{0.02} &       0.0 &  \textbf{0.01} &      -0.0 \\
\bottomrule
\end{tabular}

\caption[Accuracy increase over LSA for $\alpha=0.01$ on TREC datasets]{Accuracy increase over LSA for $\alpha=0.01$ on TREC datasets}
\label{tab:batch:results:trec0.01}
\end{center}
\end{table}


\begin{table}[h]
\begin{center}

\begin{tabular}{ll|rrrr}
\toprule
   &   &   CR &  MPQA &   MR &  SUBJ \\
scheme & lsa &        &        &        &        \\
\midrule
None & 200 &     -0.01 & \textbf{0.01} & \textbf{0.02} &      0.0 \\
   & 300 &      0.0 &      0.0 &     -0.01 &     -0.01 \\
   & 400 & \textbf{0.01} &     -0.01 & \textbf{0.01} & \textbf{0.01} \\
tfchi2 & 200 & \textbf{0.02} &      0.0 &     -0.02 &      0.0 \\
   & 300 &     -0.01 &      0.0 &     -0.01 &      0.0 \\
   & 400 &     -0.01 &     -0.0 &     -0.02 &      0.0 \\
tfgr & 200 &     -0.01 & \textbf{0.01} &     -0.01 &      0.0 \\
   & 300 &     -0.0 &      0.0 & \textbf{0.01} &     -0.01 \\
   & 400 &      0.0 &      0.0 &     -0.01 & \textbf{0.02} \\
tfidf & 200 & \textbf{0.01} & \textbf{0.01} &      0.0 &      0.0 \\
   & 300 &     -0.01 & \textbf{0.01} & \textbf{0.01} &     -0.01 \\
   & 400 & \textbf{0.01} &      0.0 &     -0.01 & \textbf{0.01} \\
tfig & 200 & \textbf{0.01} & \textbf{0.01} &     -0.01 &     -0.0 \\
   & 300 &     -0.0 &      0.0 & \textbf{0.01} & \textbf{0.01} \\
   & 400 & \textbf{0.02} &     -0.0 &      0.0 &     -0.02 \\
tfor & 200 & \textbf{0.01} &     -0.0 &     -0.0 &      0.0 \\
   & 300 &     -0.02 &      0.0 &     -0.01 &     -0.01 \\
   & 400 &      0.0 &      0.0 &     -0.01 & \textbf{0.02} \\
tfrf & 200 & \textbf{0.02} &     -0.0 &     -0.01 &      0.0 \\
   & 300 &      0.0 &      0.0 &      0.0 &      0.0 \\
   & 400 &     -0.01 &     -0.0 &     -0.0 &      0.0 \\
\bottomrule
\end{tabular}

\caption[Accuracy increase over LSA for $\alpha=0.001$]{Accuracy increase over LSA for $\alpha=0.001$}
\label{tab:batch:results0.001}
\end{center}
\end{table}






\begin{table}[h]
\begin{center}

\begin{tabular}{ll|rrrrrr}
\toprule
   &   & ABBR & DESC & ENTY & HUM & LOC & NUM \\
scheme & lsa &         &         &         &         &         &         \\
\midrule
None & 200 &       -0.0 &      -0.01 &       0.0 &       0.0 &      -0.0 &      -0.01 \\
   & 300 &       0.0 &      -0.01 &       0.0 &      -0.0 &       0.0 &      -0.01 \\
   & 400 &       0.0 &  \textbf{0.01} &      -0.01 &       0.0 &       0.0 &      -0.0 \\
tfchi2 & 200 &       0.0 &  \textbf{0.01} &       -0.0 &  \textbf{0.01} &       0.0 &  \textbf{0.02} \\
   & 300 &       -0.0 &      -0.01 &       -0.0 &       0.0 &      -0.0 &       0.0 \\
   & 400 &       -0.0 &       0.0 &       0.0 &       0.0 &  \textbf{0.01} &  \textbf{0.01} \\
tfgr & 200 &       -0.0 &      -0.01 &       -0.0 &       0.0 &      -0.0 &      -0.0 \\
   & 300 &       0.0 &  \textbf{0.01} &       -0.0 &  \textbf{0.01} &       0.0 &      -0.01 \\
   & 400 &       0.0 &      -0.01 &       -0.0 &      -0.01 &  \textbf{0.01} &      -0.01 \\
tfidf & 200 &       -0.0 &       -0.0 &       -0.0 &       0.0 &       0.0 &       0.0 \\
   & 300 &       0.0 &       -0.0 &  \textbf{0.01} &      -0.01 &      -0.01 &       0.0 \\
   & 400 &       -0.0 &       -0.0 &  \textbf{0.01} &      -0.0 &      -0.0 &       0.0 \\
tfig & 200 &       0.0 &       -0.0 &       -0.0 &      -0.0 &  \textbf{0.01} &  \textbf{0.01} \\
   & 300 &       0.0 &       -0.0 &       -0.0 &  \textbf{0.01} &       0.0 &      -0.0 \\
   & 400 &       -0.0 &      -0.01 &       0.0 &      -0.01 &      -0.0 &      -0.0 \\
tfor & 200 &       -0.0 &       -0.0 &       0.0 &  \textbf{0.01} &       0.0 &      -0.01 \\
   & 300 &       0.0 &       -0.0 &       0.0 &      -0.01 &      -0.0 &      -0.01 \\
   & 400 &       -0.0 &       0.0 &  \textbf{0.01} &      -0.0 &       0.0 &      -0.01 \\
tfrf & 200 &       0.0 &  \textbf{0.05} &  \textbf{0.01} &  \textbf{0.01} &      -0.01 &  \textbf{0.01} \\
   & 300 &       0.0 &  \textbf{0.03} &       -0.0 &       0.0 &       0.0 &       0.0 \\
   & 400 &       0.0 &  \textbf{0.03} &  \textbf{0.02} &       0.0 &      -0.01 &      -0.0 \\
\bottomrule
\end{tabular}

\caption[Accuracy increase over LSA for $\alpha=0.001$ on TREC datasets]{Accuracy increase over LSA for $\alpha=0.001$ on TREC datasets}
\label{tab:batch:results:trec0.001}
\end{center}
\end{table}




\end{document}






