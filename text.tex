\chapter*{Úvod}
\addcontentsline{toc}{chapter}{Úvod}

Slovné vektory sú jedným zo spôsobov reprezentácie slov v~algoritmoch strojového učenia a~spracovania prirodzeného jazyka. Zvyčajne sú predstavované mnohorozmernými vektormi reálnych čísel. Ich úlohou je reprezentovať syntaktické a~sémantické vzťahy medzi slovami. Slová, ktoré sú si syntakticky alebo sémanticky podobné, by zároveň mali mať podobné aj prislúchajúce vektorové reprezentácie. Takéto slovné vektory sa následne používajú v~ďalších metódach strojového učenia, napríklad strojovom preklade alebo rozoznávaní sentimentu. V~poslednej dobe sa rozvíja vytváranie slovných vektorov aj s~dôrazom na morfológiu (tvaroslovie) slov. Kým pôvodné metódy brali slovo ako nedeliteľný celok, morfologické metódy sa pozerajú aj na jednotlivé časti slova (predpony, prípony a~pod.). 

Cieľom našej práce je vyvinúť novú metódu tvorby slovných vektorov, ktorá bude klásť dôraz na morfológiu jednotlivých slov. Zároveň bude mať porovnateľnú úspešnosť s~existujúcimi metódami, bude schopná predpovedať slovné vektory pre nové slová a~bude menej pamäťovo náročná. Nevýhodou už existujúcich metód je totiž nutnosť vytvárania a~pamätania si vektorov pre všetky slová v~dostupnom korpuse. Väčšina z~týchto metód tiež nevie vytvárať slovné vektory pre nikdy nevidené slová. Náš model by mohol byť schopný rýchlejšie sa učiť (vďaka nízkej pamäťovej náročnosti), a~zároveň by mohol lepšie fungovať na špecializovaných korpusoch. 

Hlavná časť našej práce sa zaoberá návrhom nového modelu pre slovné vektory založeného na skladaní ngramov, ktorý zodpovedá vyššie uvedeným požiadavkám. Následne predstavujeme metódy na analýzu tohto modelu a~spôsoby, akým sa dá ďalej znížiť jeho pamäťová náročnosť. 

Prvá kapitola tvorí úvod do problematiky strojového učenia a~neurónových sietí, ktoré sú potrebné pri vytváraní modelov slovných vektorov. Ďalej v~nej predstavujeme rôzne spôsoby strojovej reprezentácie slov a~známy model Word2Vec. Ten je základom všetkých modelov vytváraných v~našej práci. V~druhej kapitole prinášame prehľad publikovaných prác zaoberajúcich sa slovnými vektormi s~využitím morfológie. Tretia kapitola opisuje návrh, implementáciu a~testovanie našich modelov. V~štvrtej kapitole sa zaoberáme metódami analýzy nami navrhnutého modelu, a~zároveň ponúkame spôsob, ako ešte viac znížiť jeho pamäťovú náročnosť. 
