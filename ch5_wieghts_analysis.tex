\chapter{Weights analysis} \label{chap:weight:analysis}

    \section{Interpretability}
    Most machine learning algorithm are black box approaches \cite{ribeiro2016should}. % explainability
    It is hard, to justify their predictions or to extract some insight from them. 
    
    We designed our approach with this in mind.
    Because our approach just re-weights words, we argue that it is explainable and we can extract important insight about our data from it.
    In this section we present some of the extracted insights for our datasets.
    
    Detailed weight analysis and qualitative comparison of different weighting schemes will be presented in chapter \ref{chap:weight:analysis}
    
    \* %TODO write chapter 5

    \section{Effect of reweighting}
    % weights w efective increase the number of apparances of given word and incentive LSA to not forget about this word.
    % It is hard rigorously quantify this, we will examine it in qualitative manner
    % we look on which words were boosted and which were inhibited.
    
    % we effectively start with some weighted scheme and we based on the weighs we see, which words were undervaluated and which were overvaluated.
    
    \subsection{Datasets}
    
    
    \subsubsection{CCR}
    \subsubsection{MPQA}
    \subsubsection{MR}
    \subsubsection{SUBJ}
    

    \section{Raw weight analysis}
    
    
    \section{Schemes comparison}
